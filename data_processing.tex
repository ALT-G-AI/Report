\section{Data Processing}
\label{sec:data_processing}

  \subsection{Word2Vec}
  \label{sec:word2vec}
  Words must be embedded into a numerical space for the majority of learning methods. Word2Vec uses a two-layer neural network to autoencode the words against their neighbours in the input corpus. Word2Vec produces word encodings in large vector-spaces. Semantic analogies are preserved in the space, (i.e. $queen - king + man = woman$). We trained Word2Vec on our own corpus to embed words for most of our algorithms, however, for our windowed deep neural network, we used the `Global Vectors for Word Representation` (GloVe) algorithm using a pre-trained vector of 6B words from wikipedia. These two methods are compared in Table \ref{tab:window_res}.

  \subsection{Sentence Transformations}
  \label{sec:sentence_transformations}

    It is necessary to transform the variable-length sentences to a fixed length
    representation. Two methods have been used throughout the project: padding
    (or truncating) sentences to known length using a special padding symbol and
    producing windows into a sentence. Several windows can be gained from each
    sentence: augmenting the dataset.

    For example, the sentence \textit{``This process, however, afforded me no
      means of ascertaining the dimensions of my dungeon [...]''} would produce windows
    such as \texttt{['this', 'process', ',', 'however', ','], ['process', ',',
      'however', ',', 'afforded'], [',', 'however', ',', 'afforded', 'me'], ...}

    Two encodings for the class labels have been experimented with: one hot
    encoding and assigning integers to each class.
